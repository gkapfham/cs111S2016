% CS 111 style
% Typical usage (all UPPERCASE items are optional):
%       \input 111pre
%       \begin{document}
%       \MYTITLE{Title of document, e.g., Lab 1\\Due ...}
%       \MYHEADERS{short title}{other running head, e.g., due date}
%       \PURPOSE{Description of purpose}
%       \SUMMARY{Very short overview of assignment}
%       \DETAILS{Detailed description}
%         \SUBHEAD{if needed} ...
%         \SUBHEAD{if needed} ...
%          ...
%       \HANDIN{What to hand in and how}
%       \begin{checklist}
%       \item ...
%       \end{checklist}
% There is no need to include a "\documentstyle."
% However, there should be an "\end{document}."
%
%===========================================================
\documentclass[11pt,twoside,titlepage]{article}
%%NEED TO ADD epsf!!
\usepackage{threeparttop}
\usepackage{graphicx}
\usepackage{latexsym}
\usepackage{color}
\usepackage{listings}
\usepackage{fancyvrb}
%\usepackage{pgf,pgfarrows,pgfnodes,pgfautomata,pgfheaps,pgfshade}
\usepackage{tikz}
\usepackage[normalem]{ulem}
\tikzset{
    %Define standard arrow tip
%    >=stealth',
    %Define style for boxes
    oval/.style={
           rectangle,
           rounded corners,
           draw=black, very thick,
           text width=6.5em,
           minimum height=2em,
           text centered},
    % Define arrow style
    arr/.style={
           ->,
           thick,
           shorten <=2pt,
           shorten >=2pt,}
}
\usepackage[noend]{algorithmic}
\usepackage[noend]{algorithm}
\newcommand{\bfor}{{\bf for\ }}
\newcommand{\bthen}{{\bf then\ }}
\newcommand{\bwhile}{{\bf while\ }}
\newcommand{\btrue}{{\bf true\ }}
\newcommand{\bfalse}{{\bf false\ }}
\newcommand{\bto}{{\bf to\ }}
\newcommand{\bdo}{{\bf do\ }}
\newcommand{\bif}{{\bf if\ }}
\newcommand{\belse}{{\bf else\ }}
\newcommand{\band}{{\bf and\ }}
\newcommand{\breturn}{{\bf return\ }}
\newcommand{\mod}{{\rm mod}}
\renewcommand{\algorithmiccomment}[1]{$\rhd$ #1}
\newenvironment{checklist}{\par\noindent\hspace{-.25in}{\bf Checklist:}\renewcommand{\labelitemi}{$\Box$}%
\begin{itemize}}{\end{itemize}}
\pagestyle{threepartheadings}
\usepackage{url}
\usepackage{wrapfig}
\usepackage{hyperref}
%=========================
% One-inch margins everywhere
%=========================
\setlength{\topmargin}{0in}
\setlength{\textheight}{8.5in}
\setlength{\oddsidemargin}{0in}
\setlength{\evensidemargin}{0in}
\setlength{\textwidth}{6.5in}
%===============================
%===============================
% Macro for document title:
%===============================
\newcommand{\MYTITLE}[1]%
   {\begin{center}
     \begin{center}
     \bf
     CMPSC 111\\Introduction to Computer Science I\\
     Spring 2016\\
     \medskip
     \end{center}
     \bf
     #1
     \end{center}
}
%================================
% Macro for headings:
%================================
\newcommand{\MYHEADERS}[2]%
   {\lhead{#1}
    \rhead{#2}
    \immediate\write16{}
    \immediate\write16{DATE OF HANDOUT?}
    \read16 to \dateofhandout
    \lfoot{\sc Handed out on \dateofhandout}
    \immediate\write16{}
    \immediate\write16{HANDOUT NUMBER?}
    \read16 to\handoutnum
    \rfoot{Handout \handoutnum}
   }

%================================
% Macro for bold italic:
%================================
\newcommand{\bit}[1]{{\textit{\textbf{#1}}}}

%=========================
% Non-zero paragraph skips.
%=========================
\setlength{\parskip}{1ex}

%=========================
% Create various environments:
%=========================
\newcommand{\PURPOSE}{\par\noindent\hspace{-.25in}{\bf Purpose:\ }}
\newcommand{\SUMMARY}{\par\noindent\hspace{-.25in}{\bf Summary:\ }}
\newcommand{\DETAILS}{\par\noindent\hspace{-.25in}{\bf Details:\ }}
\newcommand{\HANDIN}{\par\noindent\hspace{-.25in}{\bf Hand in:\ }}
\newcommand{\SUBHEAD}[1]{\bigskip\par\noindent\hspace{-.1in}{\sc #1}\\}
%\newenvironment{CHECKLIST}{\begin{itemize}}{\end{itemize}}

\begin{document}

\MYTITLE{Lab 5\\Assigned: February 24, 2016\\
Due: March 2, 2016 by 2:30pm }

% \vspace{-0.05in}
\subsection*{Objectives}
\vspace{-0.05in}

In addition to enhancing your teamwork skills, to write a Java program that manipulates strings of Deoxyribonucleic acid
(DNA) by appropriately using methods from the {\tt java.util.String} and {\tt java.util.Random} classes. Additionally,
to explore the fundamental approaches to using Java classes and methods to organize a solution to an interdisciplinary
problem.

\vspace{-0.05in}
\subsection*{Important Notes}
\vspace{-0.05in}

This is your second team-based assignment. As in the past assignment, you must work in a team of two during the
laboratory session and throughout the coming week. You may select your own partner, as long as you are not working with
the same person from the previous assignment.  As this is the most challenging assignment so far this semester, you and
your partner should plan your time this week accordingly and work on it incrementally. Please make sure that you use
Slack and your team's Git repository to collaborate effectively. Remember, divide and conquer!

\vspace{-0.05in}
\subsection*{General Guidelines for Labs}
\vspace{-0.05in}

\begin{itemize}

  \itemsep 0in

\item
{\bf Work on the Alden Hall computers.} If you want to work on a different
machine, be sure to transfer your programs to the Alden
machines and re-run them before submitting.
\item
{\bf Update your repository often!} You should add, commit,
and push your updated files each time you work on them.  I will not grade
your programs until the due date has passed.
\item
{\bf Review the Honor Code policy.} You
may discuss programs with others, but programs that are nearly identical
to others will be taken as evidence of violating the Honor Code.
\end{itemize}

\vspace{-0.2in}
\subsection*{Reading Assignment}
\vspace{-0.05in}

To learn more about Java strings and random numbers, review Sections 3.1--3.5 in your textbook. To learn more about how
to create and use Java classes, you may also want to read Sections 4.1 and 4.2, focusing on accepted conventions for
organizing a class. Finally, you should carefully study the sample program described in the section in this assignment
called ``{Study A Sample Program}''.

\vspace{-0.05in}
\subsection*{Create a New Directory and a Java Program}
\vspace{-0.05in}

You and your partner are responsible for implementing a complete solution to a real-world problem involving the analysis
of DNA strings in the field of bioinformatics. To start, you and your partner should create a new Git repository hosted
by Bitbucket. This new repository must adhere to the naming convention {\tt
<first-partner-username>-<second-partner-username>-cs111S2016-lab6}. After creating the repository with the correct
name, please share it with each other and with the course instructor. Now, make sure that you can both access this
repository through Git commands such as {\tt clone}, {\tt pull}, {\tt add}, {\tt commit}, and {\tt push}. Additionally,
you should make your own {\tt lab6/} directory in your personal repository and make sure that, by the end of the
laboratory assignment, all of the files that you collaboratively created are also available in this directory. Please
see the course instructor or one of the teaching assistants if you cannot complete these first two steps.

% See the section called {\tt DNA Manipulation} in this document for the specifics on what you need to do in your {\tt Lab5.java} program.

\vspace{-0.05in}
\subsection*{Study a Sample Program}
\vspace{-0.05in}

Go to the shared course repository and pull {\tt StringDemo.java}. Copy this program into the {\tt lab6/} directory
inside your own {\tt cs111S2016-<your user name>} repository.  Open this program and examine it to see what it does.
Then run it a few times with different input strings.\\

\noindent{\bf Questions to Discuss with Your Group Member}
(not to be handed in):
\begin{itemize}
\item Where is the random number generator {\em declared}?
\item Suppose you name the random number generator {\tt wilbur} instead of {\tt r}.
What other lines in the program need to be changed? Why do you have to change these lines?
\item Where is the scanner {\em declared}?
\item Suppose you name the scanner {\tt orville} instead of {\tt scan}.
What other lines in the program need to be changed? Why do you have to change these lines?
\item
Suppose the line:

\vspace*{-.2in}
\begin{center}
\verb$s1 = s1.toUpperCase();$
\end{center}
\vspace*{-.2in}

is changed to:

\vspace*{-.2in}
\begin{center}
\verb$s1.toUpperCase();$
\end{center}

Will the program still correctly display the upper-case version of {\tt s1}?
\item
Do the statements {\tt s1.toUpperCase();} and {\tt s1.toLowerCase();}
change the contents of {\tt s1}? (See previous question for more details about this matter!)
\item
Suppose the user types {\tt abcde} when asked to enter a string.
\begin{itemize}
\item Write this down with numbers to indicate the positions (indices) of each character.
\item
What position number is immediately to the left of the letter {\tt a}?
\item
What position number is immediately to the right of the letter {\tt e}?
\item
How many different ways are there to insert an ``{\tt x}'' into the string
{\tt abcde}? List all the position numbers where this {\tt x} could be placed.
(Don't forget the beginning and end.)
\item
According to the book, the {\tt nextInt(num)} method of the {\tt Random}
class returns a random number in the range 0 to {\tt num - 1}.
List all the values that could be returned by ``{\tt r.nextInt(5)}''. Please ask the course instructor if you cannot
make this list.
\item
Answer the question in the comment next to the statement ``\verb$location = r.nextInt(len+1);$''.
\end{itemize}
\item
On paper, write down the string {\tt ABCDEFG}, with position numbers
above the letters.
\begin{itemize}
\item
Underline the portion corresponding to the
expression:
\begin{center}
\verb$"ABCDEFG".substring(0,3)$
\end{center}
\item
Underline the portion corresponding to the
expression:
\begin{center}
\verb$"ABCDEFG".substring(3)$
\end{center}
\item
What string do we get if we evaluate the expression:
\begin{center}
\verb$"ABCDEFG".substring(0,3) + "ABCDEFG".substring(3)$
\end{center}
\end{itemize}
Explain, in English words, what the following statement does:
\begin{center}
\verb$s2 = s1.substring(0,location) + `x' + s1.substring(location);$
\end{center}
\item
What is the value of the expression \verb$"PQRST".charAt(0)$ ?
\item
What is the value of the expression \verb$"PQRST".charAt(2)$ ?
\item
What is the value of the expression \verb$"PQRST".charAt(4)$ ?
\item
List all possible values that can be returned by the expression
\verb$r.nextInt(5)$
\item
Explain, in English words, what the following statement does:
\begin{center}
\verb$c = "PQRST".charAt(r.nextInt(5));$
\end{center}
\end{itemize}

\noindent Don't continue with the assignment until you understand the answers to all of these questions!  You can discuss
these questions with your partner, the teaching assistants, and your instructor.

\vspace{-0.05in}
\subsection*{Creating a DNA Manipulation Program}
\vspace{-0.05in}

Bioinformatics is the study of biological phenomena by the use of biology, mathematics and computer science. One of the
most important study areas in bioinformatics concerns DNA. Deoxyribonucleic acid (DNA) is a molecule that encodes the
genetic instructions (genes) which are used by all known living organisms and many viruses to build the proteins
required to sustain existence. The genes of DNA are written in the nucleotides; guanine (G), adenine (A), thymine (T),
and cytosine (C), (chemical compounds) which serve as the alphabet of the genetic language.  Essentially, a DNA string
is a string consisting of only the letters {\tt A}, {\tt C}, {\tt G}, and {\tt T}, for instance, ``{\tt CAATGTCAC}''.
These strings encode various genetic traits such as hair color, eye color, and many others.

Each DNA string has a {\em complement} formed by replacing each code letter by its complementary code. {\tt A} and {\tt
T} are complements; so are {\tt G} and {\tt C}. Thus, the complement to the string ``{\tt CAATGTCAC}'' is ``{\tt
GTTACAGTG}''.  DNA sometimes undergoes a {\em mutation}. There are three types of mutation: insertion of a new letter
somewhere in the string; removal of a letter from the string; and replacement of one letter by another. The following
table shows the examples of the replacement of letters and the complement of the given sequence. Do you understand what
took place during each of these replacements? Make sure that you explain the meaning of each row in the table to your
partner. Of course, you may talk with the course instructor or a teaching assistant or post a question in our Slack team
if you find these examples to be difficult to understand.

\begin{center}

\begin{tabular}{|cc|}
\hline
\textbf{Strand} & \textbf{Sequence}\\
\hline\hline
$S$      & $\mathtt{ACGTGCCTCTTGGTAC}$ \\
\hline
A $\to$ T &  $\mathtt{TCGTGCCTCTTGGTTC}$ \\
T $\to$ A &  $\mathtt{TCGAGCCACAAGGATC}$ \\
C $\to$ G &  $\mathtt{TGGAGGGAGAAGGATG}$ \\
G $\to$ C &  $\mathtt{TGCACGGAGAACCATG}$ \\
\hline
$S_{complementary}$ &  $\mathtt{TGCACGGAGAACCATG}$ \\
\hline
\end{tabular}
\end{center}

You and your partner should design and implement a Java program that does the following.  Note that all changes are made
to the {\em original} input string; they are not cumulative.

\begin{enumerate}
\item
Ask the user to type in a string of DNA, declare a variable and save the DNA string that the user inputs into a variable called ``{\tt dnaString}''.

\item Print the complement of {\tt dnaString}, appropriately labeled.  (Hint: use several applications of the String
  class's ``{\tt replace}'' method. For example, if I want to replace all characters `A' with characters `T' in the
  String variable {\tt dnaString}, then I will say {\tt dnaString.replace(`A',`T');}, but there's still a trick in this
  case of getting the complement!)

Another hint regarding the trick: As you make your substitutions to get your complementary string, remember that you are
replacing, for example, \emph{A}'s to \emph{T}'s from the $S$ to make our complementary sequence ($S_{complementary}$).
Do not simply perform such substitution directly because you will be unable to figure out which \emph{T}'s were original
\emph{T}'s (and not the replaced ones) that you were supposed to change to \emph{A}'s. You can check your complementary
sequence from the website  (using the \emph{Complement} option after you enter your sequence): \url{http://arep.med.harvard.edu/labgc/adnan/projects/Utilities/revcomp.html}.

\item[NOTE:]

The next three parts of your program will also use Java's {\tt java.util.Random} class. You may want to read about the
{\tt Random} class at the bottom of this document to understand the concepts behind the {\tt Random} class better; ask
your course instructor for help with this class.

\item
Perform a random mutation consisting of inserting a randomly-chosen
extra letter into {\tt dnaString}; it
must be one of the four allowed letters. Print this, appropriately labeled
and identifying the position of the insertion and the letter inserted.
\item
Perform a random mutation consisting of removing a letter from a
randomly-chosen position in {\tt dnaString}; print this, appropriately labeled
and identifying the position of the insertion and the letter removed.
\item
Perform a random mutation consisting of altering a single letter from a
randomly-chosen position in {\tt dnaString}; it must be changed to a
randomly-chosen letter from the set of allowed letters. Print it, appropriately
labeled
and identifying the position of the replacement, the new letter, and the
letter it replaces.
\end{enumerate}

The following gives two sample runs of a previously implemented version of the program. You will get different values, of course, since the
changes are random, but the structure of the output will be the same. Please see the course instructor if you do not
understand this program's output.

\begin{verbatim}
     aldenv5:lab6 jjumadinova$ java ManipulateDNA
     Janyl Jumadinova
     Lab 6
     Thu Sep 30 12:51:58 EDT 2015

     Enter a string containing only C, G, T, and A: actg
     Complement of ACTG is TGAC
     Inserting T at position 0 gives TACTG
     Deleting from position 1 gives ATG
     Changing position 2 gives ACGG
\end{verbatim}

\begin{verbatim}
     aldenv5:lab6 jjumadinova$ java ManipulateDNA
     Janyl Jumadinova
     Lab 6
     Thu Sep 30 12:52:58 EDT 2015

     Enter a string containing only C, G, T, and A: actg
     Complement of ACTG is TGAC
     Inserting G at position 0 gives GACTG
     Deleting from position 0 gives CTG
     Changing position 0 gives ACTG
\end{verbatim}

In the second example, nothing was changed in the last line---the {\tt ManipulateDNA} program randomly replaced the
letter ``{\tt A}'' with the letter ``{\tt A}''!  This behavior is acceptable.

\vspace{-0.05in}
\subsection*{Additional Program Requirements}
\vspace{-0.05in}
\begin{itemize}

\item As in past assignments, make sure your program prints the names of both of your team members, the lab number, and
  the date as the first few output lines in the terminal.

\item Make sure your program contains the comment header with the honor pledge, the names of all team members, lab
  number, date, and the purpose of the program.

\item Make sure you document your program properly, by using comments throughout your program whenever appropriate; you
  and your partner should both understand the code's comments.

\item Make sure your output in the terminal window is neat (e.g., no missing spaces) and that the source code of your
  Java program is precisely formatted (e.g, indenting and blank spaces).

\end{itemize}

\vspace{-0.25in}
\subsection*{Summary of the Required Deliverables}
\vspace{-0.05in}

In addition to turning in printed and signed versions, for this assignment you are invited to submit electronic versions
of the following deliverables through the Bitbucket repository that you created for your team. As you complete this
step, you should make sure that you created a {\tt lab6/} directory within this Git repository.  Then, you can save all
of the required deliverables in the {\tt lab6/} directory---please see the course instructor or a teaching assistant if
you are not able to create your directory properly. Please make sure that, along with keeping the files in your team's
Git repository, that you also save and commit the file to your individual course repository; this will help you as you
are reviewing for upcoming quizzes and examinations in this course.

\vspace*{-.1in}
\begin{enumerate}

  \itemsep0in

  \item A completed, properly commented and formatted {\tt ManipulateDNA.java} program.

  \item An output document containing at least \textbf{three different} outputs obtained after running {\tt
    ManipulateDNA} in the terminal window at least three times.

  \item A document describing the strategy your team developed for completing this assignment and the work each team
    member has completed. While the {\tt ManipulateDNA} program and the output file have to be the same for both members
    of the same team, each of you should forthrightly include any challenges that you individually faced in this document.

\end{enumerate}
\vspace{-0.1in}

Share your program, the output file and the document describing your team work with me through your Git repository by
correctly using ``{\tt git add}'', ``{\tt git commit}'', and ``{\tt git push}'' commands. When you are done, please
ensure that the Bitbucket Web site has a {\tt lab6/} directory in your repository with the two files called {\tt
ManipulateDNA.java}, {\tt output}, and {\tt report}. You should see the course instructor if you have questions about
assignment submission.

\vspace{-0.05in}
\subsection*{A Quick Tutorial on Random Number Generation}
\vspace{-0.05in}
To generate random numbers, we need an object of the {\tt Random} class.
You can name this variable anything you want---``{\tt rand}'' or ``{\tt
random}'' are good names. To create a new random number generator
named {\tt rand}, be sure to import the {\tt Random} class:

\vspace{-0.1in}
\begin{center}
\verb$ import java.util.Random;$
\end{center}
\vspace{-0.1in}

and then create an instance of this class:

\vspace{-0.1in}
\begin{center}
\verb$Random rand = new Random();$
\end{center}
\vspace{-0.1in}

The three most useful methods in the {\tt Random} class are {\tt nextInt}, {\tt nextFloat}, and {\tt nextDouble}. If
{\tt rand} is the name of our random number generator (it can be called something else) and {\tt n} is a positive
integer, {\tt rand.nextInt(n)} produces an {\tt int} in the range 0, \ldots, {\tt n}-1 and {\tt rand.nextDouble()} and
{\tt rand.nextFloat()} produce a {\tt double} and a {\tt float} respectively in the range from 0 to 1 (not including 1).
By being clever, we can get different ranges---here are a few examples. In the Java code, assume {\tt i} is an {\tt int}
variable, {\tt d} is a {\tt double} variable, and {\tt rand} is an object of the {\tt Random} class.

\begin{center}
\begin{tabular}{p{2.5in}p{3.5in}}
\multicolumn{1}{c}{\bf Desired Range} & \multicolumn{1}{c}{\bf Java
Statement}\\\hline
0, 1, 2, 3, 4, 5 & \verb$i = rand.nextInt(6);$\\
10, 11, \ldots, 19, 20 & \verb$i = rand.nextInt(11) + 10;$\\
-5, -4, -3, \ldots, 4, 5 & \verb$i = rand.nextInt(11) - 5;$\\
0, 3, 6, 9, 12 & \verb$i = 3 * rand.nextInt(5);$\\
-1, 1 & \verb$i = 2 * rand.nextInt(2) - 1;$\\
$0 \leq d < 1$ & \verb$d = rand.nextDouble();$\\
$0 \leq d < 10$ & \verb$d = 10 * rand.nextDouble();$\\
$-5 \leq d < 5$ & \verb$d = 10 * rand.nextDouble() - 5;$\\
0.0, 0.1, 0.2, \ldots, 0.9, 1.0 & \verb$d = rand.nextInt(11)/10.0;$
\end{tabular}
\end{center}

% You can also use random numbers to do other creative things, for example pick out characters randomly. Let's say I want
% to pick out a random letter out of a first name, I can first create and initialize a variable {\tt{String professorName =
% "Janyl";}} Then I can generate a random number between $0$ and $4$ and save it into a variable of type {\tt int} as
% {\tt{ int randomNumber = rand.nextInt(5)}}. Finally, I  use {\tt{charAt}} method to select the character at a position
% created by the random number generator as {\tt{professorName.charAt(randomNumber)}}.

\end{document}
