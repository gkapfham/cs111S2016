% CS 111 style
% Typical usage (all UPPERCASE items are optional):
%       \input 111pre
%       \begin{document}
%       \MYTITLE{Title of document, e.g., Lab 1\\Due ...}
%       \MYHEADERS{short title}{other running head, e.g., due date}
%       \PURPOSE{Description of purpose}
%       \SUMMARY{Very short overview of assignment}
%       \DETAILS{Detailed description}
%         \SUBHEAD{if needed} ...
%         \SUBHEAD{if needed} ...
%          ...
%       \HANDIN{What to hand in and how}
%       \begin{checklist}
%       \item ...
%       \end{checklist}
% There is no need to include a "\documentstyle."
% However, there should be an "\end{document}."
%
%===========================================================
\documentclass[11pt,twoside,titlepage]{article}
%%NEED TO ADD epsf!!
\usepackage{threeparttop}
\usepackage{graphicx}
\usepackage{latexsym}
\usepackage{color}
\usepackage{listings}
\usepackage{fancyvrb}
%\usepackage{pgf,pgfarrows,pgfnodes,pgfautomata,pgfheaps,pgfshade}
\usepackage{tikz}
\usepackage[normalem]{ulem}
\tikzset{
    %Define standard arrow tip
%    >=stealth',
    %Define style for boxes
    oval/.style={
           rectangle,
           rounded corners,
           draw=black, very thick,
           text width=6.5em,
           minimum height=2em,
           text centered},
    % Define arrow style
    arr/.style={
           ->,
           thick,
           shorten <=2pt,
           shorten >=2pt,}
}
\usepackage[noend]{algorithmic}
\usepackage[noend]{algorithm}
\newcommand{\bfor}{{\bf for\ }}
\newcommand{\bthen}{{\bf then\ }}
\newcommand{\bwhile}{{\bf while\ }}
\newcommand{\btrue}{{\bf true\ }}
\newcommand{\bfalse}{{\bf false\ }}
\newcommand{\bto}{{\bf to\ }}
\newcommand{\bdo}{{\bf do\ }}
\newcommand{\bif}{{\bf if\ }}
\newcommand{\belse}{{\bf else\ }}
\newcommand{\band}{{\bf and\ }}
\newcommand{\breturn}{{\bf return\ }}
\newcommand{\mod}{{\rm mod}}
\renewcommand{\algorithmiccomment}[1]{$\rhd$ #1}
\newenvironment{checklist}{\par\noindent\hspace{-.25in}{\bf Checklist:}\renewcommand{\labelitemi}{$\Box$}%
\begin{itemize}}{\end{itemize}}
\pagestyle{threepartheadings}
\usepackage{url}
\usepackage{wrapfig}
\usepackage{hyperref}
%=========================
% One-inch margins everywhere
%=========================
\setlength{\topmargin}{0in}
\setlength{\textheight}{8.5in}
\setlength{\oddsidemargin}{0in}
\setlength{\evensidemargin}{0in}
\setlength{\textwidth}{6.5in}
%===============================
%===============================
% Macro for document title:
%===============================
\newcommand{\MYTITLE}[1]%
   {\begin{center}
     \begin{center}
     \bf
     CMPSC 111\\Introduction to Computer Science I\\
     Spring 2016\\
     \medskip
     \end{center}
     \bf
     #1
     \end{center}
}
%================================
% Macro for headings:
%================================
\newcommand{\MYHEADERS}[2]%
   {\lhead{#1}
    \rhead{#2}
    \immediate\write16{}
    \immediate\write16{DATE OF HANDOUT?}
    \read16 to \dateofhandout
    \lfoot{\sc Handed out on \dateofhandout}
    \immediate\write16{}
    \immediate\write16{HANDOUT NUMBER?}
    \read16 to\handoutnum
    \rfoot{Handout \handoutnum}
   }

%================================
% Macro for bold italic:
%================================
\newcommand{\bit}[1]{{\textit{\textbf{#1}}}}

%=========================
% Non-zero paragraph skips.
%=========================
\setlength{\parskip}{1ex}

%=========================
% Create various environments:
%=========================
\newcommand{\PURPOSE}{\par\noindent\hspace{-.25in}{\bf Purpose:\ }}
\newcommand{\SUMMARY}{\par\noindent\hspace{-.25in}{\bf Summary:\ }}
\newcommand{\DETAILS}{\par\noindent\hspace{-.25in}{\bf Details:\ }}
\newcommand{\HANDIN}{\par\noindent\hspace{-.25in}{\bf Hand in:\ }}
\newcommand{\SUBHEAD}[1]{\bigskip\par\noindent\hspace{-.1in}{\sc #1}\\}
%\newenvironment{CHECKLIST}{\begin{itemize}}{\end{itemize}}

\begin{document}

\MYTITLE{Lab 4 \\Assigned: February 10, 2016\\Due: Wednesday, February 17, 2016 by 2:30 pm}

\subsection*{Objectives}
\vspace{-0.05in}

To gain more experience using graphics methods in Java; to learn to creatively use appropriate methods to produce a
stunning piece of art; to get familiar with how two Java classes may interact. Interested students may consider
enhancing the final version of their artwork if they want to upload it to the popular Web site and affiliated apps for
\url{http://simpledesktops.com/}.

\vspace{-0.1in}
\subsection*{General Guidelines for Labs}
\vspace{-0.05in}
\begin{itemize}
\item
{\bf Work on the Alden Hall computers.} If you want to work on a different
machine, be sure to transfer your programs to the Alden
machines and re-run them before submitting.
\item
{\bf Update your repository often!} You should add, commit,
and push your updated files each time you work on them.  I will not grade
your programs until the due date has passed.
\item
{\bf Review the Honor Code policy.} You
may discuss programs with others, but programs that are nearly identical
to others will be taken as evidence of violating the Honor Code.
\end{itemize}

\vspace{-0.15in}
\subsection*{Reading Assignment}
\vspace{-0.05in}

To learn more about Java graphics, please review Sections 2.7--2.9 in your textbook, paying particular attention to the
material about the coordinate system, the specification of RGB color values, and the creation of a Java applet. Students
who want to learn more about drawing shapes should review the program that we discussed in class and study the source
code example in Listing 2.11.

\vspace{-0.05in}
\subsection*{Obtain the Given Template Programs}
\vspace{-0.05in}

In the course's ``share'' repository, after you type ``{\tt git pull}'' command, go to the {\tt lab4/} directory, where you
will find two Java programs: {\tt Lab4Display.java} and {\tt Lab4.java}. Using the graphical file browser or the
terminal window, copy the {\tt lab4/} directory from the ``share'' repository to your own {\tt cs111F2015-<your user
name>} repository inside the {\tt labs/} directory.

\subsection*{Using Java to Create a Masterpiece}

In the {\tt Lab4Display.java} program, you need to add a comment header with the Honor Code pledge, your name, date, lab
number, and the purpose of the program. Then, you need to modify the line that starts with ``{\tt JFrame window}'' by
printing your own name. This is all you need to do for this Java class! You should also note that you may change the size of
the window by modifying the width and the height in the ``{\tt window.setSize(600, 400);}'' line.

In the {\tt Lab4.java} program you will create your drawing. First, add a comment header to this class as well.  Then,
try to come up with some interesting, yet simple, image ideas: a cactus in a pot, a hat on a face, a fish in the sea, or
something more innovative.  Then try to figure out a way to draw them using only rectangles, ovals, arcs, and straight
lines.  Finally, try to express your idea using Java methods like {\tt fillRect}, {\tt drawOval}, {\tt fillArc}, and
{\tt drawLine}. Notice that the name of the graphics object is called {\tt page}, so to use a method {\tt fillRect} you
will need do something like: ``{\tt page.fillRect(10,10,50,50)}'' as shown in the book.  You may find it helpful to create
your drawing on paper first! Please see the instructor or a teaching assistant if you have a question.

\vspace*{-.1in}
\subsubsection*{Program Requirements}

\begin{itemize}

  \item The {\tt paint} method in the {\tt Lab4} class must have at least \textbf{ten} objects (these could be things
    like rectangles, ovals, arcs, strings, or lines), of at least \textbf{three} different types.

  \item Your drawing should be a concrete representation of something, it cannot be randomly placed rectangles and ovals
    without a clear meaning that you articulate in the program's comments.

  \item To ensure that it is easy to maintain, your program should declare and use integer variables to keep track of
    the location of the object(s) instead of only using numeric literals.

\end{itemize}

\begin{sloppypar}
Remember, to compile both files ({\tt Lab4.java} and {\tt Lab4Display.java}), you may use the ``{\tt javac *.java}''
command to compile all the {\.java} files in your directory. You only need to run {\tt Lab4Display.java} --- since it is
the one that contains the {\tt main} method --- and then look for the pop-up window with a Java symbol. You should
compile and run your programs incrementally after drawing each object, instead of waiting until you finish the commands
for your entire drawing.
\end{sloppypar}

\vspace*{-.1in}
\subsection*{Required Deliverables}

In addition to submitting signed and printed versions of all materials, for this assignment you are invited to submit
electronic versions of the following deliverables through the Bitbucket repository. As you complete this step, you
should make sure that you created a {\tt lab4/} directory within the Git repository.  Then, you can save all of the
required deliverables in the {\tt lab4/} directory --- please see the course instructor or a teaching assistant if you are
not able to create your directory properly.

\begin{enumerate}

        \item A completed, properly commented and formatted {\tt Lab4.java} and {\tt Lab4Display.java} program. Please
          make sure that your Java programs include the comment header file with the Honor Code, your name, date, and
          the description of the program.

        \item An output (your drawing) from running {\tt Lab4Display} in the terminal window. You can take a screenshot
          of your output, paste it into another software application (e.g., LibreOffice or gimp) and save it. Please see
          the instructor or a teaching assistant if you have questions about creating an electronic version of your
          output or saving it in Bitbucket.

\end{enumerate}

\vspace{-0.1in}

Share your program and the output file with me through your Git repository by correctly using ``{\tt git add}'', ``{\tt
git commit}'', and ``{\tt git push}'' commands. When you are done, please ensure that the Bitbucket Web site has a {\tt
lab4/} directory in your repository with only the three files called {\tt Lab4.java}, {\tt Lab4Display.java}, and {\tt
output}. You should review the ``Git Cheatsheet'' and see the course instructor if you have questions about assignment
submission using the {\tt git} commands.

\end{document}
